On considère les fonctions $f$ et $g$ suivantes: 

\[f :\: t \longmapsto  4t + 3\quad \text{et}\quad  g :\: t \longmapsto 6t.\]

Leurs représentations graphiques $\left(d_1\right)$ et $\left(d_2\right)$ sont tracées ci-dessous.

\begin{center}
\psset{xunit=2.25cm,yunit=0.4cm,comma=true}
\begin{pspicture}(-1,-3)(4.1,24)
\multido{\n=-1.0+0.1}{52}{\psline[linewidth=0.1pt](\n,-3)(\n,24)}
\multido{\n=-1.0+0.5}{11}{\psline[linewidth=0.7pt](\n,-3)(\n,24)}
\multido{\n=-3+1}{28}{\psline[linewidth=0.1pt](-1,\n)(4.1,\n)}
\multido{\n=0+5}{5}{\psline[linewidth=0.7pt](-1,\n)(4.1,\n)}
\psaxes[linewidth=1.25pt,Dy=5,Dx=0.5]{->}(0,0)(-1,-3)(4.1,24)
\psaxes[linewidth=1.25pt,Dy=5,Dx=0.5](0,0)(-1,-3)(4.1,24)
\psplot[plotpoints=2000,linecolor=red,linewidth=1.25pt]{0}{4}{6 x mul}
\psplot[plotpoints=2000,linecolor=blue,linewidth=1.25pt]{-0.75}{4}{4 x mul 3 add}
\uput[d](3.5,17){\blue $\left(d_2\right)$}\uput[u](3,18){\red $\left(d_1\right)$}
\uput[dr](0,0){O}
\rput(-0.75,-2){\blue Départ}\rput(-0.75,-3){\blue Camille}
\rput(0,-2){\red Départ}\rput(0,-3){\red Claude}
\end{pspicture}
\end{center}

\medskip

\begin{enumerate}
\item %Associer chaque droite à la fonction qu'elle représente.
$\left(d_1\right)$ est la représentation d'une fonction linéaire donc de la fonction $g$ ; effectivement $g(1) = 6$.

Donc $\left(d_2\right)$ la représentation d'une fonction  affine $f$ ; effectivement $f(2) = 4 \times 2 + 3 = 11$.
\item %Résoudre par la méthode de votre choix l'équation $f(t) = g(t).

$\bullet~~$\emph{Graphiquement} : on voit que les deux droites sont sécantes en (1,5~;~9). On a donc $S = \{1,5\}$.

$\bullet~~$\emph{Par le calcul} : $f(t) = g(t)$ soit $4t + 3 = 6t$ d'où en ajoutant $-4t$ à chaque membre : 

$3 = 2t$ et en multipliant chaque membre par $\dfrac{1}{2}$ : \: $\dfrac{3}{2} = 1,5 = t$.
\end{enumerate}

%Camille et Claude décident de faire exactement la même randonnée mais Camille part $45$~min avant Claude. On sait que Camille marche à la vitesse constante de $4$ km/h et Claude marche à la vitesse constante de $6$~km/h.

\smallskip

\begin{enumerate}[resume]
\item %Au moment du départ de Claude, quelle est la distance déjà parcourue par Camille ?

Camille a marché pendant 45 min soit $\dfrac{45}{60} = \dfrac{3\times 15}{4 \times 15} = \dfrac{3}{4}$~(h).

Elle a donc parcouru : $4 \times \dfrac{3}{4} = 4 \times 3 \times \dfrac{1}{4} = 3$~(km).
\end{enumerate}

On note $t$ le temps écoulé, exprimé en heure, depuis le départ de Claude. Ainsi $t = 0$ correspond au moment du départ de Claude.

\begin{enumerate}[resume]
\item %Expliquer pourquoi la distance en kilomètre parcourue par Camille en fonction de $t$ peut s'écrire $4t + 3$.

La distance parcourue par Camille est proportionnelle à sa vitesse soit 4~(km/h), mais pour $t = 0$, elle a déjà parcouru 3~km, donc la distance parcourue à partir du moment où Claude démarre est $3 + 4t = 4t + 3 = f(t)$.
\item %Déterminer le temps que mettra Claude pour rattraper Camille.

La distance parcourue par Claude est proportionnelle à sa vitesse 6~(km/h), donc égale à 

$6t = g(t)$.

Claude rattrape Camille quand ils sont à la même distance du départ, donc au point commun aux deux droites (question 2.) donc au bout de 1,5~h soit 1 h 30~min à 9 km du départ.
\end{enumerate}

%\begin{center}
%\begin{pspicture}[showgrid=true](0,0)(4,4)
%\psset{PtNameMath=false}
%\psset{dotscale=0.5}
%\psset{PointSymbol=*}\footnotesize
%\pstEllipse[Options] (O)(a, b)[angleA] [angleB]
%\def\ra{2.4}\def\rb{0.8} 
%\pstGeonode[PosAngle=-90,PointNameSep=0.2](2,2){O} %
%\psellipse[linecolor=red!60](O)(\ra,\rb) \pstEllipse[linecolor=red!60](O)(\ra,\rb)[0][120]
%\pstEllipse[linecolor=green!60,linestyle=dashed,arrows=->, arrowscale=1.2](O)(\ra,\rb)[120][200]
%\pstEllipse[linecolor=blue!60](O)(\ra,\rb)[200][300] 
%\pstEllipse[linecolor=purple!60,linestyle=dashed,arrows=->,arrowscale=1.2](O)(\ra,\rb)[300][360] \pstEllipse[linecolor=cyan!60](O)(\rb,\ra) 
%\end{pspicture}
%\end{center}
