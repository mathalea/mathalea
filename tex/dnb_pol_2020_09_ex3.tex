\parbox{0.7\linewidth}{On considère le motif initial ci-contre.

Il est composé d'un carré ABCE de côté $5$~cm et d'un triangle EDC, rectangle et isocèle en D.}
\hfill
\parbox{0.28\linewidth}{\psset{unit=1cm}
\begin{pspicture}(3,4)
\pspolygon[fillstyle=solid,fillcolor=lightgray](0.5,0.5)(2.5,0.5)(2.5,2.5)(1.5,3.5)(0.5,2.5)%ABCDE
\uput[dl](0.5,0.5){A}\uput[dr](2.5,0.5){B}\uput[ur](2.5,2.5){C}\uput[u](1.5,3.5){D}\uput[ul](0.5,2.5){E}
\psline(2.5,2.5)(0.5,2.5)
\end{pspicture}}

\bigskip

\textbf{Partie 1}

\medskip

\begin{enumerate}
\item Donner, sans justification, les mesures des angles $\widehat{\text{DEC}}$  et $\widehat{\text{DCE}}$.
\item Montrer que le côté [DE] mesure environ $3,5$~cm au dixième de centimètre près.
\item Calculer l'aire du motif initial. Donner une valeur approchée au centimètre carré près.
\end{enumerate}

\bigskip

\textbf{Partie 2}

\medskip

\parbox{0.4\linewidth}{On réalise un pavage du plan en partant du motif initial et en utilisant différentes transformations du plan.

Dans chacun des quatre cas suivants, donner sans justifier une transformation du plan qui permet de passer :

\begin{enumerate}
\item Du motif 1 au motif 2
\item Du motif 1 au motif 3
\item Du motif 1 au motif 4
\item Du motif 2 au motif 3
\end{enumerate}}
\hfill
\parbox{0.57\linewidth}{
\def\motifa{
\psset{unit=1cm}
\begin{pspicture}(2.1,3.5)
\psline[fillstyle=solid,fillcolor=lightgray](0,0)(2.1,0)(2.1,2.1)(1.05,3.22)(0,2.1)(0,0)
\end{pspicture}
}%ABCDE
\def\motifb
{\psset{unit=1cm}
\begin{pspicture}(2.1,3.5)
\psline(0,0)(2.1,0)(2.1,2.1)(1.05,3.22)(0,2.1)(0,0)
\end{pspicture}
}
\psset{unit=0.75cm}
\begin{pspicture}(11,10)
\psline(0,0)(4,0)(5,1)(6,0)(8,0)(8,2)(6,2)(5,1)(4,2)(2,2)(2,0)
\psline(8,2)(10,2)(11,3)(10,4)(8,4)(10,4)(10,6)(8,6)(7,5)(8,4)(8,2)
\psline(8,6)(8,8)(6,8)(4,8)(4,10)(2,10)(2,8)(3,7)(4,8)
\psline(6,8)(6,6)(7,5)(6,4)(6,2)
\psline(6,6)(4,6)(3,7)(2,6)(0,6)(0,8)(2,8)
\psline(0,0)(0,2)(1,3)(2,2)(4,2)(4,4)(6,4)
\psline(1,3)(2,4)(4,4)(2,4)(2,6)
\psline(0,6)(0,4)(1,3)
\pspolygon[linewidth=2pt,fillstyle=solid,fillcolor=lightgray](2,4)(2,6)(3,7)(4,6)(4,4)
\pspolygon[linewidth=2pt,fillstyle=solid,fillcolor=lightgray](4,6)(4,4)(6,4)(7,5)(6,6)
\pspolygon[linewidth=2pt,fillstyle=solid,fillcolor=lightgray](4,2)(4,4)(6,4)(6,2)(5,1)
\pspolygon[linewidth=2pt,fillstyle=solid,fillcolor=lightgray](6,2)(6,4)(7,5)(8,4)(8,2)
\uput[l](2,4){A} \uput[dl](4,4){B} \uput[ur](4,6){C} \uput[u](3,7){D} 
\uput[ul](2,6){E} \uput[ur](6,6){F} \uput[dr](6,4){G} \uput[u](7,5){H} 
\uput[ur](8,4){I} \uput[dr](8,2){J} \uput[dr](6,2){K} \uput[d](5,1){L} 
\uput[dl](4,2){M}
\rput(3,5){motif 1} \rput(5,5){motif 2}\rput(7,3){motif 3}\rput(5,3){motif 4}
\end{pspicture}
}

\bigskip

\textbf{Partie 3}

\medskip

Suite à un agrandissement de rapport $\dfrac{3}{2}$ de la taille du motif initial, on obtient un motif agrandi.

\medskip

\begin{enumerate}
\item Construire en vraie grandeur le motif agrandi.
\item Par quel coefficient doit-on multiplier l'aire du motif initial pour obtenir l'aire du motif agrandi?
\end{enumerate}