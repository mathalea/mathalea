%%%%%%%%%%%%%%%%%%%%%%%%%%%%%%%%%%%%%%%%%%%%
%%% Préambule pour le fichier Coop-Maths %%%
%%%%%%%%%%%%%%%%%%%%%%%%%%%%%%%%%%%%%%%%%%%%



% Remarques : 
%		* changements importants le 28 juillet 2018 pour faciliter la lecture. La police Myriade Pro est utilisée.
%
%		* il faut penser à  doubler l'interligne le plus souvent possible avec \begin{spacing}{2} ou \begin{enumerate}[itemsep=1em]
%
%

% Prérequis : 
%		* Compiler avec XeTeX
%		* Avoir le répertoire images avec les entêtes, pieds de pages et icones
%
%
% Utilisation : 
%		* La fiche commence avec \theme{nombres|gestion|grandeurs|geo|algo}{Texte (entrainement, évaluation, mise en route...}{numéro de version ou vide}{titre du thême et niveau}
%
%		* On peut préférer \themelight{couleur_theme}{couleur_numerotation} notamment pour faire des corrigés à  décalquer
%
%
%
%		* Chaque exercice commence par \exo{titre ou vide}. Le numéro de l'exercice utilise le compteur de section
%		
%		* \begin{correction}...\end{correction} remet le compteur d'exercice à  0, met le cadre adéquat et change la couleur du thême en vert 
%
%		* Plusieurs environnements sont définis : objectif, methode (dans lequel il faut faire précéder le titre d'un \iconemethode
%
%		* Une macro \cfbox permet de faire la même chose que \fbox mais avec la couleur du thême
%
%		* Une macro \reference pour donner la référence d'un objectif dans la couleur du thême. Par exemple \reference{G01}
%
%       * Une macro \motimportant qui met en gras dans la couleur du thême.
%
%		* Une macro \mathimportant qui met en gras dans la couleur du thême à  l'intérieur d'un mode math.
%
%		* Une macro \mathimportantcorr qui met en gras dans la couleur de la correction à  l'intérieur d'un mode math.
%

%		* Une macro \pointillés qui met des pointillés sur la longueur choisie
%
%		* Une macro \point{x}{y}{nom} qui trace un point dans le repêre au couleur du thême
%		
%		* Une macro \pointpos{x}{y}{nom}{position} qui trace un point dans le repêre au couleur du thême
%
%		* Une macro \pointcorr{x}{y}{nom} qui fait la même chose dans la couleur de la correction
%
%		* Une macro \pointcorrpos{x}{y}{nom}{position} qui trace un point dans le repêre au couleur du thême
%
%		* Une macro \tampon(nom) qui insêre le graphique tampon-nom en haut à  droite de la feuille
%
%	* Une macro \version(v) qui insêre le numéro de version en haut à  droite de la feuille (à  ne pas utiliser en même temps que tampon)
%
%		* Une macro \mathunderline pour souligner en couleur correction une partie d'une expression mathématique
%
%	* Une macro \deg pour le signe des degrés
%
%	* Une macro \exalea{id} qui met l'image du dé et le lien en gras vers l'exercice 
%

\usepackage[left=1.5cm,right=1.5cm,top=3.5cm,bottom=2cm]{geometry}
%\usepackage[utf8]{inputenc}		        % Accents, encodage utf8
% Inutile avec XeTeX ?
%\usepackage[T1]{fontenc}		        		% Encodage des caractêres
\usepackage{lmodern}			        		% Choix de la fonte (Latin Modern de D. Knuth)
\usepackage[francais]{babel}	        		% Les rêgles de typo. franà§aises
\usepackage{multicol} 					% Multi-colonnes
\usepackage{calc} 						% Calculs 
\usepackage{enumerate}					% Pour modifier les numérotations
\usepackage{enumitem}
\usepackage{graphicx}					% Pour insérer des images
\usepackage{tabularx}					% Pour faire des tableaux
\usepackage{pgf,tikz}					% Pour les images et figures géométriques
\usetikzlibrary{arrows,calc,fit,patterns,plotmarks,shapes.geometric,shapes.misc,shapes.symbols,shapes.arrows,
shapes.callouts, shapes.multipart, shapes.gates.logic.US,shapes.gates.logic.IEC, er, automata,backgrounds,chains,topaths,trees,petri,mindmap,matrix, calendar, folding,fadings,through,positioning,scopes,decorations.fractals,decorations.shapes,decorations.text,decorations.pathmorphing,decorations.pathreplacing,decorations.footprints,decorations.markings,shadows} % Charge toutes les librairies de Tikz
\usepackage{tkz-tab,tkz-fct,tkz-euclide}	% Géométrie euclidienne avec TikZ
\usetkzobj{all}				
\usepackage{amsmath,amsfonts,amssymb,mathrsfs}  % Spécial math
\usepackage[squaren]{SIunits}			% Pour les unités (gêre le conflits avec  \square de l'extension amssymb)
\usepackage{pifont}						% Pour les symboles "ding"
\usepackage{bbding}						% Pour les symboles
\usepackage[misc]{ifsym}						% Pour les symboles
\usepackage{cancel}						% Pour pouvoir barrer les nombres
\usepackage{url} 			        		% Pour afficher correctement les url
 
\usepackage{eurosym}						% Pour utiliser la commande \euro
\usepackage{fancyhdr,lastpage}          	% En-têtes et pieds
 \pagestyle{fancy}                      	% de pages personnalisés
\usepackage{fancybox}					% Pour les encadrés
\usepackage{xlop}						% Pour les calculs posés
\usepackage{standalone}					% Pour avoir un apercu d'un fichier qui sera utilisé avec un input
\usepackage{multido}						% Pour faire des boucles
\usepackage[hidelinks]{hyperref}					% Pour gérer les liens hyper-texte
\renewcommand{\url}[1]{\textcolor{couleur_theme}{\href{http://#1}{#1}}} 	% Pour ajouter automatiquement le http
\usepackage{fourier}
\usepackage{colortbl} 					% Pour des tableaux en couleur
\usepackage{setspace}					% Pour \begin{spacing}{2.0} \end{spacing}
\usepackage{multirow}					% Pour des cellules multilignes dans un tableau
\usepackage{import}						% Equivalent de input mais en spécifiant le répertoire de travail
\usepackage[]{qrcode}					% Pour la commande \qrcode
\usepackage{pdflscape}
\usepackage[framemethod=tikz]{mdframed} % Pour les cadres
\usepackage{tikzsymbols}
\usepackage{tasks}						% Pour les listes horizontales


\graphicspath{{./images/}}				% Le répertoire


%%% Choix de la police


\usepackage{mathspec}
\setmainfont{Myriad Pro}
\setmathrm{Myriad Pro}
\setmathsf{Myriad Pro}
\setmathtt{Myriad Pro}
%\setboldmathrm[BoldFont={Optima ExtraBlack}]{Optima Bold}
\setmathfont(Digits){Myriad Pro}
\setmathfont(Latin){Palatino}

\spaceskip=2\fontdimen2\font plus 3\fontdimen3\font minus3\fontdimen4\font\relax %Pour doubler l'espace entre les mots

%\doublespacing %double l'interligne



%\usepackage{fontspec}
%\usepackage{mathspec}

%\setmainfont{Raleway}

%\setallmainfonts(Digits,Latin){Raleway}

%\setmathrm{latinmodern}
%\setmathfont(Digits,Latin){latinmodern}

%%% Mise en forme

\setlength{\parindent}{0mm}				% Pas de retrait en début de paragraphe
\setlength\multicolsep{0pt} 				% Pour que l'environnement multicols ne commence pas par un saut de ligne
\renewcommand{\arraystretch}{1.5}		% Interligne dans les tableaux
\renewcommand{\labelenumi}{\textbf{\theenumi{}.}}		% Numérotation en gras
\renewcommand{\labelenumii}{\textbf{\theenumii{}.}}		% Numérotation de niveau 2 en gras
\renewcommand{\thesection}{\Roman{section}.}				% Numérotation des sections en chiffres romains
\renewcommand{\thesubsection}{\alph{subsection})}		% Numérotation des sous-sections en lettres
\setlength{\columnsep}{1cm}								% Séparation des colonnes
\setlength{\columnseprule}{1.5pt}
\renewcommand{\columnseprulecolor}{\color{couleur_theme}} % Trait de séparation des colonnes en couleur


\setlength\arrayrulewidth{1.5pt} 	% Epaisseur des les filets des tableaux
\arrayrulecolor{couleur_theme}		% Couleur des filets des tableaux


\renewcommand{\headrulewidth}{0pt} % Pour enlever les traits en en-tête et en pied de page
\renewcommand{\footrulewidth}{0pt}
\fancyhead[L]{}
\fancyhead[R]{}



%%% Couleurs

\definecolor{nombres}{cmyk}{0,.8,.95,0}
\definecolor{gestion}{cmyk}{.75,1,.11,.12}
\definecolor{gestionbis}{cmyk}{.75,1,.11,.12}
\definecolor{grandeurs}{cmyk}{.02,.44,1,0}
\definecolor{geo}{cmyk}{.62,.1,0,0}
\definecolor{algo}{cmyk}{.69,.02,.36,0}
\definecolor{correction}{cmyk}{.63,.23,.93,.06}



%%% Environnements


\newmdenv[linecolor=couleur_theme, linewidth=3pt,topline=true,rightline=false,bottomline=false,frametitlerule=false,frametitlefont={\color{couleur_theme}\bfseries},frametitlerulewidth=1pt]{methode}


\newmdenv[startcode={\setlength{\multicolsep}{0cm}\setlength{\columnsep}{.2cm}\setlength{\columnseprule}{0pt}\vspace{0cm}},linecolor=white, linewidth=3pt,innerbottommargin=10pt,innertopmargin=5pt,innerrightmargin=20pt,splittopskip=20pt,splitbottomskip=10pt,everyline=true,tikzsetting={draw=couleur_theme,line width=4pt,dashed,dash pattern= on 10pt off 10pt},frametitleaboveskip=-.6cm,frametitle={\tikz\node[anchor= east,rectangle,fill=white]{\textcolor{couleur_theme}{\raisebox{-.3\height}{\includegraphics[width=.8cm]{\iconeobjectif}}\; \bfseries \Large Objectifs}};}]{objectif}

\newmdenv[startcode={\colorlet{couleur_numerotation}{correction}\renewcommand{\columnseprulecolor}{\color{correction}}
\setcounter{section}{0}\arrayrulecolor{correction}},linecolor=white, linewidth=4pt,innerbottommargin=10pt,innertopmargin=5pt,splittopskip=20pt,splitbottomskip=10pt,everyline=true,frametitle=correction,tikzsetting={draw=correction,line width=3pt,dashed,dash pattern= on 15pt off 10pt},frametitleaboveskip=-.4cm,frametitle={\tikz\node[anchor= east,rectangle,fill=white]{\; \textcolor{correction}{\raisebox{-.3\height}{\includegraphics[width=.6cm]{icone-correction}}\; \bfseries \Large Corrections}};}]{correction}

\newmdenv[roundcorner=0,linewidth=0pt,frametitlerule=false, backgroundcolor=gray!40,leftmargin=8cm]{remarque}



%%% Macros



\newcommand{\theme}[4]
{
	\fancyhead[C]{
	\begin{tikzpicture}[remember picture,overlay]
	  \node[anchor=north east,inner sep=0pt] at ($(current page.north east)+(0,-.8cm)$) {\includegraphics{header-#1}};
	  \node[anchor=east, fill=white] at ($(current page.north east)+(-2,-1.4cm)$) {\Huge \textcolor{couleur_theme}{\bfseries \#} #2 \textcolor{couleur_theme}{\bfseries \MakeUppercase{#3}}};
	  \node[anchor=center, color=white] at ($(current page.north)+(0,-2.65cm)$) {\Large \bfseries \MakeUppercase{#4}};
	\end{tikzpicture}
	}
	\fancyfoot[C]{
	\begin{tikzpicture}[remember picture,overlay]
	  \node[anchor=south west,inner sep=0pt] at ($(current page.south west)+(0,0)$) {\includegraphics{footer-#1}};
	\end{tikzpicture} 
	}
	\colorlet{couleur_theme}{#1}
	\colorlet{couleur_numerotation}{couleur_theme}
	\def\iconeobjectif{icone-objectif-#1}
	\def\urliconeomethode{icone-methode-#1}
}

\newcommand{\themelight}[2]
{
\pagestyle{empty}
\colorlet{couleur_numerotation}{#2}
\colorlet{couleur_theme}{#1}
\geometry{top=1cm,bottom=1cm}

}




\newcommand{\numb}[1]{ % Dessin autour du numéro d'exercice
\begin{tikzpicture}[overlay,yshift=-.3cm,scale=.8]
	\draw[fill=couleur_numerotation,couleur_numerotation](-.3,0)rectangle(.5,.8);
	\draw[line width=.05cm,couleur_numerotation,fill=white] (0,0)--(.5,.5)--(1,0)--(.5,-.5)--cycle;
	\node[couleur_numerotation]  at (.5,0) { \large \bfseries #1};
		\draw (-.4,.8) node[white,anchor=north west]{\bfseries EX}; 
\end{tikzpicture}
}


\usepackage{titlesec} % Le titre de section est en fait un numéro d'exercice avec sa consigne alignée à  gauche.
% Avant le titre était plus gros
%\titleformat{\section}{\bfseries\large}{\numb{\arabic{section}}}{1cm}{\hspace{0em}}{}

\titleformat{\section}{}{\numb{\arabic{section}}}{1cm}{\hspace{0em}}{}




\newcommand{\exo}[1]{ % Un titre d'exercice est en f	ait une nouvelle section avec la consigne écrite en caractêres normaux
\section{\textmd{#1}}
\medskip
}

\newcommand{\iconemethode}[0]{ % Pour l'icone des méthodes dans la bonne couleur
\raisebox{-.3\height}{\includegraphics[width=1cm]{\urliconeomethode}}\quad
}


\newcommand{\cfbox}[1]{% Cadre dans la couleur du thême
	\setlength{\fboxrule}{2pt}
    \colorlet{currentcolor}{.}%
    {\color{couleur_theme}%
    \fbox{\color{currentcolor}#1}}%
}

\newcommand{\corrfbox}[1]{% Cadre dans la couleur de la correction
	\setlength{\fboxrule}{2pt}
    \colorlet{currentcolor}{.}%
    {\color{correction}%
    \fbox{\color{currentcolor}#1}}%
}


\newcommand{\reference}[1]{\ignorespaces
\textcolor{couleur_theme}{\textbf{#1 -}}\xspace
}

\newcommand{\refexo}[1]{\ignorespaces
\textcolor{couleur_theme}{\textbf{\;Ex #1}}\xspace
}

\newcommand{\motimportant}[1]{\ignorespaces
\textcolor{couleur_theme}{\textbf{#1}}\xspace
}

\newcommand{\mathimportant}[1]{
\mathbin{\color{couleur_theme}\pmb{#1}}
}

\newcommand{\mathimportantcorr}[1]{
{\color{correction}\pmb{#1}}
}

\newcommand{\motimportantcorr}[1]{\ignorespaces
\textcolor{correction}{\textbf{#1}}\xspace
}

\newcommand{\pointilles}[1]{
\makebox[#1]{\dotfill}
}

\newcommand{\point}[3]{
\draw[couleur_theme,ultra thick] (#1,#2)--++(.3,.3)--++(-.6,-.6)--++(.3,.3)--++(-.3,.3)--++(.6,-.6);
\draw[couleur_theme] (#1,#2)  node[above right=.1,black]{$#3$};
}

\newcommand{\pointpos}[4]{
\draw[couleur_theme,ultra thick] (#1,#2)--++(.3,.3)--++(-.6,-.6)--++(.3,.3)--++(-.3,.3)--++(.6,-.6);
\draw[couleur_theme] (#1,#2)  node[#4=.1,black]{$#3$};
}


\newcommand{\pointcorr}[3]{
\draw[correction,ultra thick] (#1,#2)--++(.3,.3)--++(-.6,-.6)--++(.3,.3)--++(-.3,.3)--++(.6,-.6);
\draw[correction] (#1,#2)  node[above right=.1,black]{$#3$};
}

\newcommand{\pointcorrpos}[4]{
\draw[correction,ultra thick] (#1,#2)--++(.3,.3)--++(-.6,-.6)--++(.3,.3)--++(-.3,.3)--++(.6,-.6);
\draw[correction] (#1,#2)  node[#4=.1,black]{$#3$};
}



\newcommand{\tampon}[1]{
	\fancyhead[R]{
	\begin{tikzpicture}[remember picture,overlay]
		\node[anchor=north east,inner sep=0pt] at ($(current page.north east)+(-.3,-.3cm)$) {\includegraphics[width=1.6cm]{tampon-#1}};
	\end{tikzpicture}
	}
}

\newcommand{\version}[1]{
	\fancyhead[R]{
	\begin{tikzpicture}[remember picture,overlay]
		\node[anchor=north east,inner sep=0pt] at ($(current page.north east)+(-.5,-.5cm)$) {\large \textcolor{couleur_theme}{\bfseries V#1}};
	\end{tikzpicture}
	}
}

\newcommand{\exalea}[1]{
	\raisebox{-.5\height}{\includegraphics[height=1cm]{exalea}} \textcolor{couleur_theme}{\bfseries \url{coopmaths.fr/ex#1}}
}

\newcommand{\youtube}[1]{
	\raisebox{-.3\height}{\includegraphics[height=.6cm]{youtube}} \textcolor{couleur_theme}{\bfseries \url{coopmaths.fr/video#1}}
}


\def\mathunderline#1{\color{correction}\underline{{\color{black}#1}}\color{black}}

\renewcommand{\deg}{\ensuremath{^\circ}}

% Pour les corrigés avec xlop : utilisation de la police du document.
% Toujours utile avec le passage à  Myriad ??? NON !
%\newcommand{\police}[1]{ 
%{\fontfamily{lmr}\selectfont #1}
%}
